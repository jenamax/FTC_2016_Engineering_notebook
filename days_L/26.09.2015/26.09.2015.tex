\subsubsection{26.09.2015}
	\textit{\textbf{Time frame:}} 16:00-21:30 \newline
	\textit{\textbf{Preview:}} The purpose for current meeting was to develop ideas were invented last meeting.\newline \newline
	\textit{\textbf{Modules:}}

  \begin{table}[H]
	\vspace{-2mm}
	\begin{center}
		\begin{tabular}{|p{0.2\linewidth}|p{0.7\linewidth}|p{0.1\linewidth}|}
			\hline
			Module & Solution & Label \\
			\hline
			Bucket for debris & The shape of bucket should form one stage. & bucket \\
			\hline
			Debris separator and lock for bucket & Flap above the enter. & bucket \\
			\hline
			Crank elevator & There were calculated basic parameters. & elevator \\
			\hline
			Gripper for debris & Axis with 2 rotating blades ahead of the bucket for grabbing debris. & gripper \\
			\hline
		\end{tabular}
	\end{center}
  \end{table}
  
   \newline
  \textit{\textbf{Detailed explaination:}}
  \begin{enumerate*}
  	\item Today was held a research on how to score debris into boxes in with maximal efficiency. Due to experiment it was revealed, that scoring cubes one-by-one won't allow to score a lot of elements because they will settle down randomly. It was discovered, that the best solution is to put 4 stages with 5 cubes in each one. Cubes in stage should be placed as 2+2+1. According to this researches, the shape of the bucket for debris should form one stage.
  	
  	\item Today was invevted one possible construction of separator for debris. It consists of axis with a flap above the bucket's enter, which can narrow it's height so as prevent balls from scoring. It will also prevent debris from falling out of the bucket while it's overturnded. This way, current mechanism will be a separator and a lock at one time.
  
  	
  	\item Gripper is a rotating brush for collecting debris. It will be used for faster collecting of the debris and also for retaining it in bucket (without gripper debris can freely escape the bucket when the robot moves backward). Gripper should be powered by 1 DC motor or 1-2 powerful servos to be fast and powerful enough. Using 2 blades at an angle of $180^\circ$ is the most convenient solution as it's simple to realise and it requires less space than construction with 3 blades at an angle $120^\circ$.
  	
  \end{enumerate*}
  
   \newline
  \textit{\textbf{Additional comments:}} The next meeting we need to revise all the aspects of concept and correct it.
  
\fillpage
