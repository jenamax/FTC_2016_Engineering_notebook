\subsubsection{28.09.2015}
	\textit{\textbf{Time frame:}} 17:00-21:30 \newline
	\textit{\textbf{Preview:}} The purpose for current meeting was to revise all our previous ideas and reveal weaknesses. Then correct concept.\newline \newline
	\textit{\textbf{Weak points}}

  \begin{table}[H]
	\vspace{-2mm}
	\begin{center}
		\begin{tabular}{|p{0.2\linewidth}|p{0.7\linewidth}|p{0.1\linewidth}|}
			\hline
			Weak point & Solution & Label \\
			\hline
			Gear ratio 2:1 on wheels can be not enough for climbing & It will be installed gear ratio 1:1 at first. If tests of gear 2:1 will be successful, it will be installed on the robot. & chassis \\
			\hline
			Shape of the bucket &  & bucket \\
			\hline
			Shape of the beam for scoring alpinists &  & alpinists \\
			\hline
		\end{tabular}
	\end{center}
  \end{table}
  
   \newline
  \textit{\textbf{Detailed explaination:}}
  \begin{enumerate*}
  	\item It's difficult to predict if the gear ratio 2:1 will provide robot with enough power for climbing to a second zone of a mountain before the test drive. That's why at first wheel base should be realised with gear ratio 1:1. A test model with gear ratio 2:1 should be assembled and tested independently. In case the testing of 2:1 model will be successfull, this gear ratio will be installed onto the main robot.
  	\begin{figure}[H]
  		\begin{minipage}[h]{1\linewidth}
  		  \center{\includegraphics[scale=0.2]{3Engineering/3Concept_discussing/concept_days/28.09.2015/images/01}}
  		  \caption{}
  		\end{minipage}
  	\end{figure}
  	
  	\item If the bucket would be a size of 5 cubes, it could keep no more 2 balls at once. That's not good enough as when cubes run out our robot becomes uneffective. So, the bucket's capacity should be improved. The possible solution is to improve the length with saving the prior width. For example, bucket with parameters 13 $\times$ 21 cm is enough for containing 5 balls.
  	\begin{figure}[H]
  		\begin{minipage}[h]{1\linewidth}
  		  \center{\includegraphics[scale=0.2]{3Engineering/3Concept_discussing/concept_days/28.09.2015/images/02}}
  		  \caption{}
  		\end{minipage}
  	\end{figure}
  	
  	\item Shelter for autonomus alpinists is only 5.75 inches in length, so the most accurate way to score climbers is to throw them vertically. To provide this, the axis of rotation should be placed higher than the resque beacon (fig. 1).
  	\begin{figure}[H]
  		\begin{minipage}[h]{1\linewidth}
  		  \center{\includegraphics[scale=0.2]{3Engineering/3Concept_discussing/concept_days/28.09.2015/images/02}}
  		  \caption{}
  		\end{minipage}
  	\end{figure}
  	
  \end{enumerate*}
  
   \newline
  \textit{\textbf{Additional comments:}} The next meeting we will structure ideas into a system.
  
\fillpage
