\subsubsection{26.09.2015}
	\textit{\textbf{Time frame:}} 16:00-21:30 \newline
	\textit{\textbf{Preview:}} The purpose for this meeting was to develop ideas that were proposed last meeting.\newline \newline
	\textit{\textbf{Modules:}}

  \begin{table}[H]
	\vspace{-2mm}
	\begin{center}
		\begin{tabular}{|p{0.2\linewidth}|p{0.7\linewidth}|p{0.1\linewidth}|}
			\hline
			Module & Solution & Label \\
			\hline
			Bucket for debris & The shape of bucket should form one stage. & bucket \\
			\hline
			Debris separator and lock for bucket & Flap above the enter. & bucket \\
			\hline
			Crank elevator & There were calculated basic parameters. & elevator \\
			\hline
			Gripper for debris & Axis with 2 rotating blades ahead of the bucket for grabbing debris. & gripper \\
			\hline
		\end{tabular}
	\end{center}
  \end{table}
  
   \newline
  \textit{\textbf{Detailed explaination:}}
  \begin{enumerate*}
  	\item Today we researched how to score debris into boxes with maximal efficiency. Experiments revealed that scoring cubes one-by-one won't allow to score a lot of elements because they will settle down randomly. It was discovered that the best solution is to put 4 stages with 5 cubes in each one. Cubes in each stage should be placed as 2+2+1. According to this researche, the shape of the bucket for debris should form one stage.
  	
  	\item Today was came up with one possible construction of the debris separator. It consists of an axis with a flap above the bucket's enter, which can narrow it's height as to prevent balls from scoring. It will also prevent debris from falling out of the bucket while it's overturned. This way, the mechanism will be a separator and a lock at one time.
  
  	
  	\item Intake is made with a rotating brush for collecting debris. It will be used for faster debris collection and also for retaining it in bucket (without the gripper debris can freely escape the bucket when the robot moves backward). Gripper should be powered by 1 DC motor or 1-2 powerful servos to be fast and powerful enough. Using 2 blades at an angle of $180^\circ$ is the most convenient solution as it's simple to realise and it requires less space than construction with 3 blades at an angle $120^\circ$.
  	
  \end{enumerate*}
  
   \newline
  \textit{\textbf{Additional comments:}} The next meeting we need to revise all the aspects of concept and correct it.
  
\fillpage
