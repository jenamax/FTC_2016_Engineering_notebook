
\section{Team PML 30 X} 
	Team PML 30 X was assembled in September 2014 in the Russian city of St. Petersburg from 3 novices and 2 participants with experience. Tasks and roles were distributed among the participants, and we established safety rules. In the first place the team put spreading principles of gracious professionalism to others. All decisions were made collectively inside team with discussion to find the most optimal solutions. 
	During the year we took part in many events and everywhere we have tried to attract attention to our team and encourage people to take part in FTC. Also we pursued and distributed the principles of honorable professionalism. Talking to the press, we hoped to attract more attention to our team and to the competition in general, as well as attracting sponsors. The latter was important because of the need for funds - purchasing materials and equipment costs a lot.
	Last season the team took part in the three qualifying competitions, in the regional finals, in European championship and in the World championship. In all of them we made new contacts, shared experience and provided mutual assistance to other teams. In the first qualifying rounds in Sochi we met Stuy Fission 310 from USA and maintain contact with them to this day. On regional finals, we met with a team from Romania, Auto Vortex, and keep in touch with them through Facebook. Also, there is an active group chat with a large number of Russian teams. You can find the team page in Facebook at the address https://www.facebook.com/pages/FTC-team-PML30-X.
	To increase the efficiency of our team work we used the version control system GitHub, which allows the entire team to work simultaneously on a single projects without losing files and providing easy way to resolve roblems. Also for writing technical books we been used professional typesetting system LaTeX.
	\begin{figure}[H]
		\center{\includegraphics[scale=0.2]{key_chapters/Team/images/09}}\\
	\end{figure}[H]
\fillpage

\subsubsection{Instructors}:

\begin{figure}[H]
	
	\begin{minipage}[h]{0.47\linewidth}
		Luzin Dmitry\\
		\emph{Head of Robotics Department in Phys-Math Lyceum 30, Saint-Peterburg, Russia. Main coach of FTC team.\\}
		\emph{Information: 26 years old, in robotics 6 years, in FTC 4 years.}
	\end{minipage}
	\hfill
	\begin{minipage}{0.47\linewidth}
		\center{\includegraphics[scale=0.3]{key_chapters/Team/images/07}}\\
	\end{minipage}
	\vfill
	\begin{minipage}[h]{0.47\linewidth}
		\center{\includegraphics[scale=0.35]{key_chapters/Team/images/08}}\\
	\end{minipage}
	\hfill
	\begin{minipage}{0.47\linewidth}
		Luzina Ekaterina \\
		\emph{Professor of Robotics Department in Phys-Math Lyceum 30, Saint-Peterburg, Russia. Tutor of FTC team. \\}
		\emph{Information: 26 years old, in robotics 6 years, in FTC 4 years.}
	\end{minipage}
\end{figure}

\begin{figure}[H]
	\begin{minipage}[h]{0.47\linewidth}
		\center{\includegraphics[scale=0.25]{key_chapters/Team/images/05}}\\
	\end{minipage}
	\hfill
	\begin{minipage}{0.47\linewidth}
		Fedotov Anton \\ 
		\emph{Professor of Robotics Department in Phys-Math Lyceum 30, Saint-Peterburg, Russia. Tutor of FTC team. \\}
		\emph{Information: 23 years old, in robotics 5 years, in FTC 4 years.}
	\end{minipage}	
	\vfill 
	\begin{minipage}[h]{0.47\linewidth}
		\center{\includegraphics[scale=0.1]{key_chapters/Team/images/06}}\\
	\end{minipage}
	\hfill
	\begin{minipage}{0.47\linewidth}
		Krylov Georgii \\ 
		\emph{Professor of Robotics Department in Phys-Math Lyceum 30, Saint-Peterburg, Russia. Tutor of FTC team. \\}
		\emph{Information: 18 years old, in robotics 4 years, in FTC 4 years.}
	\end{minipage}	
\end{figure}

\fillpage

\subsubsection{Team members}
\begin{figure}[H]
	\begin{minipage}[h]{0.47\linewidth}
		\center{\includegraphics[scale=0.27]{key_chapters/Team/images/04}}\\		
	\end{minipage}
	\hfill
	\begin{minipage}[h]{0.47\linewidth}
		Maksimychev Evgeny\\
		\emph{Role in team: captain, operator-2, responsible for the technic of safety,  writing of engineering notebook, developer of lift and bucket for scoring elements. \\}
		\emph{Information: 16 years old, in robotics 3 years, in FTC 2 year. \\}
		\emph{Why I chose FTC: "This is an interesting project that allows to implement some innovative solutions. In addition to the skills of designing robots, we also obtain the skills of the technical documentation and communication with colleagues which makes this competition as close to real engineering problems."}	
	\end{minipage}
\end{figure}
	\vfill 
	\begin{figure}[H]
	\begin{minipage}[h]{0.47\linewidth}
		Timur Babadzhanov\\
		\emph{Role in team: operator-1,  developer mechanism for scoring autonomous climbers\\ }
		\emph{Information: 15 years old, in robotics 2 years, in FTC 1 years. \\ } 
		\emph{Why I chose FTC:" It was recommended for me. Also I heared about previous seasons of FTC and decided that it will be interesting for me. Also I wanted to learn working with TETRIX that can be useful for my projects."}					
	\end{minipage}
	\hfill
	\begin{minipage}[h]{0.47\linewidth}
		\center{\includegraphics[scale=0.4]{key_chapters/Team/images/01}}\\
	\end{minipage}
\end{figure}
\hfill

\begin{figure}[H]	
	\begin{minipage}{0.47\linewidth}
		\center{\includegraphics[scale=0.6]{key_chapters/Team/images/10}}			
	\end{minipage}
	\hfill
	\begin{minipage}{0.47\linewidth}
		Ivan Afanasev\\
		\emph{Role in team: developer of gripper for debris\\ }
		\emph{Information: 16 years old, in robotics 2 years, in FTC 1 years. \\ } 
		\emph{Why I chose FTC:" It is a good realization of my engineering skills. I'm good at physics and programming and decided try robotics as merger of this subjects. FTC give opportunity to learn, meet with people from other contries. It is good for pupils such as I."}		
	\end{minipage}
\end{figure}
\hfill
\begin{figure}[H]	
	\begin{minipage}{0.47\linewidth}
		Victoria Loseva\\
		\emph{Role in team: developer of wheel base\\ }
		\emph{Information: 17 years old, in robotics 2 years, in FTC 1 years. \\} 
		\emph{Why I chose FTC:"I enjoy working on new and unique projects, and FTC is a great way for me to do exactly that: solving the challenging problem of building and designing a robot from scratch, as a team, is all it's about!"}			
	\end{minipage}	
	\hfill
	\begin{minipage}{0.47\linewidth}
		\center{\includegraphics[scale=0.7]{key_chapters/Team/images/11}}			
	\end{minipage}
\end{figure}
\fillpage


