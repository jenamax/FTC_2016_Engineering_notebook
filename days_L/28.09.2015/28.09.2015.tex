\subsubsection{28.09.2015}
	\textit{\textbf{Time frame:}} 17:00-21:30 \newline
	\textit{\textbf{Preview:}} The purpose for current meeting was to revise all our previous ideas and discover weaknesses, then correct our concept.\newline \newline
	\textit{\textbf{Weak points}}

  \begin{table}[H]
	\vspace{-2mm}
	\begin{center}
		\begin{tabular}{|p{0.2\linewidth}|p{0.7\linewidth}|p{0.1\linewidth}|}
			\hline
			Weak point & Solution & Label \\
			\hline
			The Gear ratio 2:1 on wheels can be not enough for climbing & It will be installed gear ratio 1:1 at first. If tests of the 2:1 gear will be successful, it will be installed on the robot. & chassis \\
			\hline
			Shape of the bucket &  & bucket \\
			\hline
		\end{tabular}
	\end{center}
  \end{table}
  
   \newline
  \textit{\textbf{Detailed explaination:}}
  \begin{enumerate*}
  	\item It's difficult to predict if the gear ratio 2:1 will provide the robot with enough power for climbing to the second zone of the mountain before a test drive. That's why at first wheel base should be realised with gear ratio 1:1. A test model with gear ratio 2:1 should be assembled and tested independently. In the case the testing of 2:1 model will be successful, this gear ratio will be installed onto the main robot.
  	
  	
  	\item If the bucket would be a size of 5 cubes, it could keep no more 2 balls at once. That's not good enough as when cubes run out our robot becomes ineffective. So, the bucket's capacity should be improved. The possible solution is to improve the length with keeping the prior width. For example, bucket with parameters 13 $\times$ 21 cm is enough to hold 5 balls.
  	
  \end{enumerate*}
  
   \newline
  \textit{\textbf{Additional comments:}} The next meeting we will structure ideas into a system.
  
\fillpage
