\subsubsection{24.09.2015}
	\textit{\textbf{Time frame:}} 17:00-21:30 \newline
	\textit{\textbf{Preview:}} The main purpose for current meeting was to figure out how the modules of the robot should look and how they will work. \newline \newline

  \begin{table}[H]
	\vspace{-2mm}
	\begin{center}
		\begin{tabular}{|p{0.2\linewidth}|p{0.7\linewidth}|p{0.1\linewidth}|}
			\hline
			Module & Solution & Label \\
			\hline
			Wheel base & 6 standard wheels with 6 DC motors. The center of gravity is between the second pair of wheels; the corner wheels are equidistant from the center of gravity. & chassis \\
			\hline
			Heaviness & Robot should be as light as possible to afford gear ratio for speed 2:1 on drive motors. & chassis \\
			\hline
			Elevator for debris & Crank elevator with one degree of freedom. & elevator \\
			\hline
			Bucket for debris & Bucket with turning cover which will close entry inside the bucket to prevent scoring elements from accidental falling out & bucket \\
			\hline
			Slopes for collecting debris & To increase collecting area on both sides of the bucket will be placed slopes & gripper \\
			\hline
			Separator for debris & Turnable beam before the bucket, which prevents balls from getting into the bucket in it's lower position (5cm from floor). & gripper \\
			\hline
			Scoring autonomous climber and pushing button & F - shaped beam & climbers + button\\ \\
			\hline
		\end{tabular}
	\end{center}
  \end{table}
  
   \newline
  \textit{\textbf{Detailed explaination:}}
  \begin{enumerate*}
  	\item The wheel base includes 3 pairs of wheels. The middle pair of wheels provides better rotation, because their direction corresponds with the tangent of the circle of rotation.
  	
  	The center of gravity should be on the crossing of lines which link opposite wheels. wheels on one side should be placed on one line. In this construction each wheel will obtain $\frac{1}{6}$ of robot's weight and moments on all wheels will be the same. 
  	
  	
  	\item Both standard TETRIX and "NeveRest 1:40" motors have moments around 10 kg/cm. The diameter of standard wheels is 10 cm. So, the moment on wheels will be $\frac{10\text{kg} \cdot \text{cm}}{5\text{cm}} \cdot n = 2n\text{kg}$ (n - number of motors). The moment required for climbing to the ramp is $10\text{kg} \cdot sin30^\circ = 5\text{kg}$. Consequently, 3 motors will be enough for driving robot of 10kg to the 1 zone of the ramp. It's possible to use 6 motors with gear ratio 2:1.
  	% % % %
  	\item The crank elevator is the most reliable construction. One rotating beam requires 1 DC motor.
  	
  	The moment of DC motor should be enough for moving bucket with 5 scoring elements at a lever of about 40-50 cm. Moment of 5 cubes (250g) is $0.25\text{kg} \cdot 50\text{cm} = 12.5\text{kg} \cdot \text{cm}$. Moment of bucket will be about 10kg as well. So, it was decided to use gear ratio 1:3 (it will increase motor's moment to 30kg*cm).
  	% % % %
  	\item Bucket will be made of PET. PET is the best variant because of it's little weight (weigt of $100\times 100 \times 0.5mm$ sheet is 7g) and flexibility. This plastic is transparent, so it will be possible to see how much debris inside it.
  	
  	The bucket will have a special cover for retaining deebris in the bucket during the turning of beam with bucket towards the goal.
  	
  	\item The slopes will be mounted to the carriage. They will be placed on both sides of the bucket entrance and will lead debris from corners of capturing area to the center. Additionally, they will protect wheels from debris (wheels can get stuck on debris).
  	
  	
  	\item For scoring climbers and pushing button in autonomous period it was decided use F-shaped beam that turn by servo. When it turn climbers fall to goal from the bucket that fixed on the top beam and bottom beam push the button.
  	
  \end{enumerate*}
  
   \newline
  \textit{\textbf{Additional comments:}} The next meeting we will continue developing concept.

\fillpage
