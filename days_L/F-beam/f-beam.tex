\subsubsection{Mechanism for scoring autonomous climbers and pushing button}
\paragraph{Plan of creating module:}	
	
	\begin{enumerate}
		\item Creating 3D model in Creo parametric.
		\item Calculation of moments, distances and angels.
		\item Testing and debugging of the mechanism
	\end{enumerate}
	
	The module turns an F-shaped beam using a servo. On the top of beam there is a bucket with climbers. After the beam turns, the bucket turns over, and the climbers go to the box. At the end of other part of the beam is a light sensor. It detects the color of a button and presses it. The robot performs this during the autonomous period.
  \begin{figure}[H]
		\begin{minipage}[h]{\linewidth}
			\center{\includegraphics[scale=0.5]{days_L/Meetings/images/08}}
			\caption{F-beam module}
		\end{minipage}
	\end{figure}
	
	\paragraph{Ideas that were looked:}
	
	\begin{enumerate}	
		\item The first idea was the following: we decided to use two beams. The two beams turn, and the light sensor on the end of each beam detects the correct button. The beam, which detects the correct button rests in place. The other beam turns up. The robot goes forward and presses the correct button. But this idea isn't good, because we have to use two servos, and this is very difficult, and the module take up too much space.
	
		\item Another idea was the following: we use one beam. The beam turns and the light sensor on the end of the beam detects the color of the right button. If this color is correct, the robot turns right and then goes forward. Otherwise the robot turns left and then goes forward. This algoritm is slower, but it's lighter, and the model takes not as much space. We decided to use this algoritm.
	\end{enumerate}
	
	\paragraph{Calculation of moments}
	
	We calculated the moments to compare it with the maximum moment of the servo. If the maximum moment of the servo is higher than the moment of module, the module will work. The maximum moment of the servo(HS-485HB) is 28 kg*mm. The moment of gravity forces using the following formulae: 
	
	$m_\text{constr} \cdot l_\text{constr} + m_\text{climbers} \cdot l_\text{climbers}$.	
	 $M_\text{constr}$ is the total mass of the module without climbers. $L_\text{constr}$ is the distance between the rotation axis and the COG of module without climber. $M_\text{climbers}$ is the total mass of the two climbers. $L_\text{climbers}$ is the distance between the rotation axis and the COG of climbers. All values we find in Creo Parametric.
	 
	 $M_\text{constr}$ is about 100 g or 0.1 kg. $L_\text{constr}$ is about 93 mm. The moment of construction is about 9.3 kg*mm. Mass of a climber is 22.7 g. Mass of two climbers is 0.0454 kg. The COG of climbers and geometrical center ar ine the same place, because the climbers are uniform. It is about 204 mm. The moment of gravity forces if the all module is 18.9 kg*mm. Safety factor is 1.5. With this factor moment of module is 28 kg*mm. The moment of module is not bigger than the maximum moment of the servo. The module will work.
	 %\begin{figure}[H]
	 %	\begin{minipage}[h]{\linewidth}
	 %		\center{\includegraphics[scale=0.7]{days_L/f-beam/images/02}}
	 %		\caption{The table of moments}
	 %	\end{minipage}
	 %\end{figure}
	 
	 \paragraph{Work of module}
	 
	 \begin{itemize}
	 
		 \item The dimensions of the bucket are 6x5.5x12 mm. The dimenension of the main beam are 20x0.5x1.5 mm.
	 
		 \item The robot must stay near the button at a distance of 16.9 cm. Also, the main beam must be in the middle of the beacon. Then the robot turns the beam 48 degrees and the climbers go down. Then the robot must go forward for a distance of 5 cm. After that, the robot turns beam 42 degrees. Then the robot detects correct button and turns in its side 9 degrees. At the end the robot goes forward and presses correct button.
	\end{itemize}
	\fillpage
